\newcommand{\aka}{\mbox{\itshape a.k.a.}\xspace}
\section{Introduction}
To study the creation and property of the quark-gluon plasma (QGP)
is one of the most important goals of high energy nuclear physics,
i.e. heavy-ion collision physics.
%QGP,
%of which the existence is predicted by the quantum chromodynamics (QCD),
%is a new state of matter under the conditions of extremely high temperature and/or baryon density.
%In QGP quarks and gluons are no longer confined and able to move freely.
The QGP can be created and studied in laboratory through high energy nucleus-nucleus (AA) collisions.
In central AA collisions, which are also called `large system',
lots of energy is deposited in a sizeable volume,
where the vacuum is heated up and the phase transition from hadron gas to QGP is expected to occur.
However in proton-proton and proton-nucleus collisions (\aka `small systems'),
the system volume is too small, and the deconfined QGP matter is not supposed to be formed.
These collisions are usually treated as the baseline for the studies of QGP in AA collisions.

With the formation of the QGP in the early stage of heavy-ion collisions,
the production yields and momentum spectra of those final-state hadrons might be
modified compared to the expectations based on the measurements in proton-proton collisions.
These QGP effects are called hot nuclear matter (HNM) effects. 

One of the HNM effects that can be studied is how the major feature of QGP,
deconfinement of quarks and gluons, influences the hadronisation mechanism.
Hadronisation, the process in which quarks/gluons are transformed into hadrons,
are quite different depending on whether a QGP is formed or not.
In principle, hadronisation should be calculated with first-principle QCD,
however phenomenological functions are widely used due to the non-perturbative nature of hadronization.
This is called fragmentation hadronisation,
and these functions are applicable no matter a QGP is formed or not.
When QGP is formed, quarks can also hadronise into hadrons through recombination with other quarks in QGP, and this is called recombination or coalescence. The recombination/coalescence explains the enhancement of the baryon-to-meson ratio at intermediate transverse momentum, and is not supposed to occur without the deconfined medium. 

The so-called `baryon enhancement' phenomena were first observed in anti-protons and pions in gold-gold collisions at RHIC in 2000s and inspired the proposal of coalescence hadronization mechanism. Thereafter similar phenomena are observed in strange and charm hadrons in AA collisions. However, the enhancement of $\Lambda_{c}^{+}/D^{0}$ ratio at intermediate transverse momentum, are also observed by ALICE experiment in proton-proton and proton-lead collisions at the nucleon-nucleon center-of-mass energy of $5.02 \mathrm{TeV}$, where no QGP is expected to be formed. The data can't be reproduced by fragmentation models but are consistent with some coalescence models, and this is named as `$\Lambda_{c}^{+}$ puzzle' and calls for deeper understanding and more precise measurement of hadronisation in proton-lead collisions. The $\Lambda_{c}^{+}/D^{0}$ in proton-lead collisions at $\sqrt{s_{\mathrm{NN}}}=5.02$~TeV also is measured with LHCb, and also shows enhancement at intermediate $p_{\mathrm{T}}$, which is described well by coalescence models.

Besides hot nuclear matter effects, the production yields and momentum spectra of final-state hadrons in heavy-ion collisions may also be affected by those mechanisms not related to the QGP, which are called cold nuclear matter effects (CNM). To determine whether QGP is formed in heavy-ion collisions, it is essential to distinguish cold nuclear matter effects and hot nuclear matter effects. These CNM effects include the nuclear modification of parton distribution function, which is called nuclear parton distribution function (nPDF), and rescatterings of final-state hadrons produced in heavy-ion collisions.

Cold nuclear matter effects can be studied through proton-nucleus ($p\mathrm{A}$) collisions where the HNM effects are supposed to be non-dominant. These collisions, with smaller size of systems compared to AA collisions and lager number of produced hadrons compared to proton-proton collisions, are ideal for the study of CNM effects. For example, the nPDFs can be probed or constrained by the production asymmetry of final-state particles between forward and backward rapidities.

Open heavy charmed baryon $\Lc$ is a good probe to nuclear matter effects in heavy-ion collisions. Due to the large charm quark mass, perturbative QCD is applicable in calculating their production cross sections in the initial hard parton scatterings. The charm quarks produced in QGP thermally or produced by rescattering of final-state hadrons can be neglected since the medium temperature is much less than the charm quark mass.

The LHCb experiment provides unique measurements of $\Lc$ production in proton-lead collisions in forward rapidities.
Designed as a detector whose primary goals is to study heavy-flavour physics in hadron collisions,
LHCb has more precise tracking system and particle identification system compared to other heavy-ion experiments,
well suitable for precise charm hadron reconstructions.
With its pseudo-rapidity coverage of $2.0\textless \eta \textless 5.0$,
very small Bjoken-$x$ regions ($\sim10^{-5}$) can be probed with open charm hadrons,
significantly beyond the coverage of other heavy-ion experiments such as ALICE and STAR,
giving unique information for constraining nPDFs.

