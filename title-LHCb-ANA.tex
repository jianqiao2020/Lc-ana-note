% $Id: title-LHCb-ANA.tex 39841 2013-07-26 10:31:08Z roldeman $
% ===============================================================================
% Purpose: LHCb-ANA Note title page template
% Author: 
% Created on: 2010-10-05
% ===============================================================================

%%%%%%%%%%%%%%%%%%%%%%%%%
%%%%%  TITLE PAGE  %%%%%%
%%%%%%%%%%%%%%%%%%%%%%%%%
\begin{titlepage}

% Header ---------------------------------------------------
\vspace*{-1.5cm}

\noindent
\begin{tabular*}{\linewidth}{lc@{\extracolsep{\fill}}r@{\extracolsep{0pt}}}
\ifthenelse{\boolean{pdflatex}}% Logo format choice
{\vspace*{-1.2cm}\mbox{\!\!\!\includegraphics[width=.14\textwidth]{figs/lhcb-logo.pdf}} & &}%
{\vspace*{-1.2cm}\mbox{\!\!\!\includegraphics[width=.12\textwidth]{figs/lhcb-logo.eps}} & &}
 \\
 & & LHCb-ANA-2019-039 \\  % ID 
 & & \today \\ % Date - Can also hardwire e.g.: 23 March 2010
 & & v1.7\\
\hline
\end{tabular*}

\vspace*{4.0cm}

% Title --------------------------------------------------
{\normalfont\bfseries\boldmath\huge
\begin{center}
% DO NOT EDIT HERE. Instead edit macro in main.tex to keep metadata correct
  \papertitle
\end{center}
}

\vspace*{2.0cm}

% Authors -------------------------------------------------
\begin{center}
% If changing to list here, make pdfauthors in main.tex a comma
% separated list with the same names. Otherwise metadata in file will be wrong.
    Chenxi Gu$^1$, Giulia Manca$^2$, Yiheng Luo$^3$,
    Jiayin Sun$^2$, Jianqiao Wang$^4$, Di Yang$^4$, Xianglei Zhu$^4$
\bigskip\\
{\normalfont\itshape\footnotesize
$ ^1$ Laboratoire Leprince-Ringuet, Palaiseau, France\\
$ ^2$Peking University, Beijing, China \\
$ ^3$Universita e INFN, Cagliari, Italy \\
$ ^4$Tsinghua University, Beijing, China
}
\end{center}

\vspace{\fill}

% Abstract -----------------------------------------------
\begin{abstract}
  \noindent
  Open heavy charmed baryon $\Lc$ is a good probe to nuclear matter effects in heavy-ion collisions.
    Due to the large charm quark mass, perturbative QCD is applicable
    in calculating their production cross sections in the initial hard parton scatterings.
    %The charm quarks produced in QGP thermally or produced by rescattering of final-state hadrons can be neglected since the medium temperature is much less than the charm quark mass.
    With the data of proton-lead collisions at $\sqsnn=8.16 \tev$
    collected by LHCb during Run 2 in 2016,
    the double-differential production cross section of $\Lc$ baryon
    is measured in the transverse momentum region of ~$2-15~\gevc$,
    corresponding to a center-of-mass rapidity region of $1.5< y^{*} < 4.0$
    for forward configurations while $-5.0 < y^{*} < -2.5$ for backward configurations.
    Open charm baryon over meson ratio $\Lc/\Dz$ is presented to study hadronisation mechanism
    and compared with results measured by ALICE in proton-proton and proton-lead collisions
    at $\sqsnn=5.02 \tev$ and results obtained by LHCb in proton-lead
    and peripheral lead-lead collisions at $\sqsnn=5.02\tev$.


\end{abstract}

\vspace*{2.0cm}
\vspace{\fill}

\end{titlepage}
\section*{History}
\begin{itemize}
    \item 0.0, Feb. $8^{\mathrm{th}}$ 2024:
\end{itemize}


\pagestyle{empty}  % no page number for the title 

%%%%%%%%%%%%%%%%%%%%%%%%%%%%%%%%
%%%%%  EOD OF TITLE PAGE  %%%%%%
%%%%%%%%%%%%%%%%%%%%%%%%%%%%%%%%

%  empty page follows the title page ----
\newpage
\setcounter{page}{2}
\mbox{~}

\cleardoublepage
